\documentclass[12pt,a4paper]{article}
\usepackage[utf8]{inputenc}
\usepackage[T2A]{fontenc}
\usepackage[russian]{babel}
\usepackage{multicol}
\usepackage{multirow}
\usepackage{float}
\usepackage{amsmath}
\usepackage{amsfonts}
\usepackage{amssymb}
\usepackage{graphicx}
\usepackage{geometry}
\geometry{left=3cm}
\geometry{right=1.5cm}
\geometry{top=2cm}
\geometry{bottom=2cm}
\title{Лабораторная работа №.}
\date{}
\pagestyle{empty}
\begin{document}
	\maketitle
	\thispagestyle{empty}
	\section{Введение}

\begin{equation}
    equation
    \label{key}
\end{equation}

\begin{equation}
    equation
\label{key}
\end{equation}


\subsection{Цель работы и решаемые задачи:}
Цель работы:
Решаемые задачи:
\begin{itemize}	
	\item ;
	\item .
\end{itemize}

\subsection{Экспериментальная установка}

\begin{figure}[h]
	\centering
	\includegraphics[width = 8cm]{yst.png}
	\caption{Экспериментальная установка}	
	\label{im:yst}	
\end{figure}

   
\section{Экспериментальные данные и их обсуждение}
Результаты измерений представлены в таблице \ref{tab:data1}. 
\begin{table}[h]
	\caption{}
	\centering
	\label{tab:data1}
	\begin{tabular}{| c | c | c | c |} \hline
 &  &  &  \\ \hline
 &  &  &  \\ \hline
   &  	&\multirow{10}{*}{15}&\multirow{10}{*}{30} \\  \cline{1-2}
	&  	&	& \\  \cline{1-2}
	&  	&	& \\  \cline{1-2}
	&  	&	& \\  \cline{1-2}
	&  	&	& \\  \cline{1-2}
	&  	&	& \\  \cline{1-2}
	&  	&	& \\  \cline{1-2}
	&  	&	& \\  \cline{1-2}
	&  	&	& \\  \cline{1-2}
	&  	&	& \\ 
		\hline
	\end{tabular}
\end{table}	

График зависимости  представлен на рисунке \ref{im:graph1}. График построен и аппроксимирован в программе Origin. 

\begin{figure}[h!]
	\centering
	\includegraphics[width = \linewidth]{graph1.png}
	\caption{График зависимости }	
	\label{im:graph1}	
\end{figure}

Результаты измерений представлены в таблице \ref{tab:data2}. 
\begin{table}[h]
	\caption{Результаты измерений }
	\centering
	\label{tab:data2}
	\begin{tabular}{| c | c | c | c | c | c |} \hline
кол-во витков & сила тока & магнитная индукция & длина катушки &\multirow{2}{*}{$\alpha = N/L$} & \multirow{2}{*}{$\Delta\alpha$} \\ \cline{1-4}
N& I, А; $\Delta I = 0{,}1$А & B, мТ;  $\Delta B = 0{,}1$мТ	& L, м; $\Delta L = 0{,}001$м & & \\ \hline
\multirow{9}{*}{30} &20 & 1,9 &	0,4	 & 75	&  0,18 \\  \cline{2-6}
&20 & 2	 &  0,36 &	83,3  & 0,2                 \\  \cline{2-6}
&20 & 2,2 &	0,32 &	93,8  & 0,3                 \\  \cline{2-6}
&20 & 2,6 &	0,28 &	107,1 &	0,4                 \\  \cline{2-6}
&20 & 3,2 &	0,24 &	125	  & 0,5                 \\  \cline{2-6}
&20 & 3,7 &	0,2	 &	150	  & 0,8                 \\  \cline{2-6}
&20 & 4,5 &	0,16 &	187,5 &	1,2                 \\  \cline{2-6}
&20 & 5,5 &	0,12 &	250	  & 2                   \\  \cline{2-6}
&20 & 6,7 &	0,08 &	375	  & 5                   \\
		\hline
	\end{tabular}
\end{table}	

\begin{figure}[h!]
	\centering
	\includegraphics[width = \linewidth]{graph2.png}
	\caption{График зависимости}	
	\label{im:graph2}	
\end{figure}


\section{Вывод}


\end{document}